%        File: hodge.tex
%     Created: mar mar 05 08:00  2019 C
% Last Change: mar mar 05 08:00  2019 C
%
\documentclass[12pt,a4paper]{article}
\usepackage[utf8]{inputenc}
\usepackage[spanish]{babel}
\usepackage{amsmath}
\usepackage{amsfonts}
\usepackage{amssymb}
\usepackage{amsthm}
\usepackage{mathtools}
\usepackage[left=2cm,right=2cm,top=2cm,bottom=2cm]{geometry}
\usepackage{tikz,tikz-cd}
\usetikzlibrary{arrows, babel}
\usepackage{url}
\usepackage[colorlinks=true,linktocpage=true,pagebackref=true,linkcolor=blue,urlcolor=blue]{hyperref}

\author{\textit{Guillermo Gallego Sánchez}}
\title{\textsc{Teoría de Hodge}}
\date{}

\newtheorem{thm}{Teorema}[section]
\newtheorem{prop}[thm]{Proposición}
\newtheorem{lema}[thm]{Lema}
\newtheorem{corol}[thm]{Corolario}
\theoremstyle{definition} \newtheorem{defn}[thm]{Definición}
\theoremstyle{definition} \newtheorem{ejemplo}[thm]{Ejemplo}
\theoremstyle{definition} \newtheorem{ejercicio}[thm]{Ejercicio}
\theoremstyle{remark} \newtheorem*{obs}{Observación}

\DeclarePairedDelimiter\norm{\lVert}{\rVert}
\DeclarePairedDelimiter\esc{\langle}{\rangle}
\newcommand{\dvol}{\mathrm{dvol}}
\newcommand{\dol}{\bar{\partial}}
\newcommand{\HH}{\mathcal{H}}
\newcommand{\RR}{\mathbb{R}}
\newcommand{\CC}{\mathbb{C}}
\newcommand{\id}{\mathrm{id}}
\newcommand{\pr}{\mathrm{pr}}
\newcommand{\NN}{\mathbb{N}}
\newcommand{\eps}{\varepsilon}
\newcommand{\bomega}{\boldsymbol{\Omega}}

\begin{document}
\maketitle
\section{La idea formal}
Sea $M$ una variedad compleja conexa y compacta de dimensión compleja $n$. Recordemos que las $k$-formas diferenciables complejas admitían una descomposición
\begin{equation*}
  \Omega^k(M)=\bigoplus_{p+q=k}\Omega^{p,q}(M)
\end{equation*}
y teníamos el siguiente bicomplejo de cocadenas
\begin{center}
  \begin{tikzcd}
    &\vdots &\vdots & \\
    \cdots \arrow{r}{\bar{\partial}}  & \Omega^{p+1,q}(M) \arrow{r}{\bar{\partial}}\arrow{u}{\partial}& \Omega^{p+1,q+1}(M)\arrow{r}{\bar{\partial}}\arrow{u}{\partial} & \cdots    \\ 
    \cdots \arrow{r}{\bar{\partial}}  & \Omega^{p,q}(M) \arrow{r}{\bar{\partial}}\arrow{u}{\partial} & \Omega^{p,q+1}(M)\arrow{u}{\partial} \arrow{r}{\bar{\partial}}& \cdots   \\ 
   &\vdots \arrow{u}{\partial}&\vdots \arrow{u}{\partial}&
  \end{tikzcd}
\end{center}
de forma que definíamos la \emph{cohomología de Dolbeaut}
\begin{equation*}
  H^{p,q}(M)=\frac{\left\{  \omega \in \Omega^{p,q}(M) : \bar{\partial}\omega=0\right\}}{\bar{\partial}(\Omega^{p,q-1}(M))}.
\end{equation*}
El propósito fundamental de la \emph{teoría de Hodge} es encontrar unos representantes «especiales» de estas clases de cohomología.
  
 Aunque las conclusiones de la teoría de Hodge no son de carácter métrico, será crucial escoger de modo auxiliar una métrica hermítica $ds^2$ en $M$. En coordenadas esta métrica puede expresarse como
 \begin{equation*}
   ds^2=\sum_{i,j} h_{ij}(z) dz_i \otimes d\bar{z}_i,
 \end{equation*}
 y por Gram-Schmidt podemos «diagonalizar» esta métrica por congruencia. Es decir, podemos encontrar una referencia \emph{unitaria}, esto es, un conjunto de formas diferenciales de tipo $(1,0)$ $\left\{ \varphi_1,\dots,\varphi_n \right\}$ tal que 
 \begin{equation*}
   ds^2=\sum_i \varphi_i \otimes \bar{\varphi}_i.
 \end{equation*}
 Con esta referencia unitaria, podemos inducir una métrica en los fibrados $\Lambda^{p,q}(T^*M)$: si 
 \begin{align*}
   \alpha_x&= \sum_{|I|=p, |J|=q} \alpha_{IJ}(x) \varphi_{i_1} \wedge \cdots \wedge \varphi_{i_p} \wedge \bar{\varphi}_{j_1} \wedge \cdots \wedge \bar{\varphi}_{j_q}|_x, \\
   \beta_x&= \sum_{|I|=p, |J|=q} \beta_{IJ}(x) \varphi_{i_1} \wedge \cdots \wedge \varphi_{i_p} \wedge \bar{\varphi}_{j_1} \wedge \cdots \wedge \bar{\varphi}_{j_q}|_x ,
 \end{align*}
  definimos
  \begin{equation*}
    \esc{\alpha_x,\beta_x}=\sum_{|I|=p, |J|=q} \overline{\alpha_{IJ}(x)} \beta_{IJ}(x)   .
  \end{equation*}
  La métrica hermítica $ds^2$ induce también una forma de volumen en $M$
  \begin{equation*}
    \dvol_M=K_n \varphi_1 \wedge \cdots \wedge \varphi_n \wedge \bar{\varphi}_1 \wedge \cdots \wedge \bar{\varphi}_n,
  \end{equation*}
  con $K_n$ cierta constante que no nos interesa.
  Definimos ahora el operador \emph{estrella de Hodge} $*:\Omega^{p,q}(M) \rightarrow \Omega^{n-p,n-q}(M)$, de forma que
  \begin{equation*}
    \alpha \wedge *\beta = \esc{\alpha,\beta} \dvol_M.
  \end{equation*}
  Usando la forma de volumen, a partir de la métrica definida en los fibrados $\esc{\cdot,\cdot}$, podemos definir una métrica «$L^2$» en los espacios de secciones $\Omega^{p,q}(M)$:
  \begin{equation*}
    \esc{\alpha,\beta}_{L^2}=\int_M \esc{\alpha,\beta} \dvol_M = \int_M \alpha \wedge *\beta.
  \end{equation*}
  Esta métrica dota a $(\Omega^{p,q}(M), \esc{\cdot,\cdot}_{L^2})$ de la estructura de espacio prehilbertiano. 

  Consideremos ahora el operador de Dolbeaut $\bar{\partial}:\Omega^{p,q}(M) \rightarrow \Omega^{p,q+1}(M)$. Este operador tiene un operador adjunto «formal» con respecto a la métrica $L^2$, $\dol^*:\Omega^{p,q+1}(M) \rightarrow \Omega^{p,q}(M)$, definido por la relación:
  \begin{equation*}
    \esc{\alpha,\dol \beta}_{L^2}=\esc{\dol^* \alpha,\beta}_{L^2},
  \end{equation*}
  para cualesquiera $\alpha \in \Omega^{p,q+1}(M)$ y $\beta \in \Omega^{p,q}(M)$.

  La idea clave ahora es la siguiente: \textit{vamos a tratar de determinar las clases de cohomología por sus elementos de norma mínima}. Es decir, nos hacemos la siguiente pregunta: dada una clase de cohomología $a\in H^{p,q}(M)$, ¿existe un representante $\alpha$ de $a$ con norma mínima?

  \begin{lema}
    Una forma $\dol$-cerrada $\alpha \in \Omega^{p,q}(M)$ es de norma mínima en $\alpha + \dol \Omega^{p,q-1}(M)$ si y sólo si $\dol^*\alpha = 0$.
  \end{lema}
  \begin{proof}
    En primer lugar, supongamos que una $\dol^*\alpha = 0$. Entonces, para toda $\beta \in \Omega^{p,q-1}(M)$ tenemos
    \begin{equation*}
      \norm{\alpha+\dol \beta}_{L^2}^2=\esc{\alpha+\dol\beta,\alpha+\dol\beta}_{L^2}=\norm{\alpha}_{L^2}^2 + \norm{\dol \beta}_{L^2}^2 + 2 \mathrm{Re} \esc{\alpha, \dol\beta}_{L^2}
    \end{equation*}
    Ahora, 
    \begin{equation*}
      \esc{\alpha,\dol \beta}_{L^2}=\esc{\dol^* \alpha, \beta}_{L^2}=0,
    \end{equation*}
    ya que $\dol^*\alpha = 0$.
    Por tanto, 
    \begin{equation*}
      \norm{\alpha+\dol \beta}_{L^2}^2=\norm{\alpha}_{L^2}^2 + \norm{\dol \beta}_{L^2}^2 \geq \norm{\alpha}_{L^2}^2.
    \end{equation*}

    Por otra parte, supongamos que $\alpha$ es de norma mínima. Entonces, para cualquier $\beta \in \Omega^{p,q-1}(M)$ ha de cumplirse que
    \begin{equation*}
      \left. \frac{\partial}{\partial t}\right|_{t=0} \norm{\alpha + t \dol \beta}^2_{L^2}=0.
    \end{equation*}
  Ahora,
  \begin{align*}
    \left. \frac{\partial}{\partial t}\right|_{t=0} \norm{\alpha + t \dol \beta}^2_{L^2}&=\left. \frac{\partial}{\partial t}\right|_{t=0} \left( \norm{\alpha}_{L^2}^2 + t^2 \norm{\beta}_{L^2}^2 + 2t \mathrm{Re}\esc{\alpha,\dol \beta}_{L^2} \right)\\ &= \left. 2t \norm{\dol \beta}^2_{L^2} + 2 \mathrm{Re}\esc{\alpha,\dol \beta}_{L^2} \right|_{t=0} = 2 \mathrm{Re}\esc{\alpha,\dol \beta}_{L^2},
  \end{align*}
  mientras que
  \begin{align*}
    \left. \frac{\partial}{\partial t}\right|_{t=0} \norm{\alpha + t \dol (i\beta)}^2_{L^2}&=\left. \frac{\partial}{\partial t}\right|_{t=0} \left( \norm{\alpha}_{L^2}^2 + t^2 \norm{\beta}_{L^2}^2 + 2t \mathrm{Im}\esc{\alpha,\dol \beta}_{L^2} \right)\\ &= \left. 2t \norm{\dol \beta}^2_{L^2} + 2 \mathrm{Im}\esc{\alpha,\dol \beta}_{L^2} \right|_{t=0} = 2 \mathrm{Im}\esc{\alpha,\dol \beta}_{L^2}.
  \end{align*}
  Tenemos entonces que $$\esc{\dol^*\alpha, \dol \beta}_{L^2}=\esc{\alpha, \dol \beta}_{L^2}=0.$$
  Como esto es cierto para cualquier $\beta \in \Omega^{p,q-1}(M)$, concluimos que $\dol^*\alpha=0$, como queríamos probar.
  \end{proof}

  De este resultado, si podemos identificar cada clase de cohomología con su elemento de norma mínima, podemos concluir, al menos formalmente, la siguiente correspondencia
  \begin{center}
    \begin{tikzcd}
      H^{p,q}(M) \arrow{r} &\left\{ \alpha \in \Omega^{p,q}(M) : \dol \alpha=0, \dol^*\alpha=0 \right\} .\arrow{l}
    \end{tikzcd}
  \end{center}
  Consideremos ahora el operador \emph{laplaciano}
  \begin{equation*}
    \Delta = \dol \dol^* + \dol^* \dol,
  \end{equation*}
  y notemos que $\Delta \alpha = 0$ si y sólo si $\dol \alpha = 0$ y $\dol^*\alpha =0$. En efecto, en un sentido es claro. Veamos entonces qué pasa si $\Delta \alpha = 0$. En tal caso, lo que sucede es que
  \begin{equation*}
    0 = \esc{\Delta \alpha, \alpha}_{L^2} = \esc{\dol \dol^* \alpha, \alpha}_{L^2} + \esc{\dol^* \dol \alpha, \alpha}_{L^2} = \esc{\dol^* \alpha, \dol^* \alpha}_{L^2} + \esc{\dol \alpha, \dol \alpha}_{L^2} = \norm{\dol^* \alpha}_{L^2} + \norm{\dol \alpha}_{L^2},
  \end{equation*}
  con lo cual, $\dol \alpha = \dol^*\alpha=0$. Definimos el conjunto de \emph{formas armónicas} de tipo $(p,q)$ como 
  \begin{equation*}
    \HH^{p,q}(M)=\left\{ \alpha \in \Omega^{p,q}(M) : \Delta \alpha=0 \right\}.   
  \end{equation*}
 Precisamente lo que queremos demostrar es lo siguiente:
 \begin{thm}[Hodge]
   Existe un isomorfismo
   \begin{center}
     \begin{tikzcd}
       H^{p,q}(M) \arrow{r}{\cong} & \HH^{p,q}(M).
     \end{tikzcd}
   \end{center}
 \end{thm}
   
 Nótese que aún no hemos probado este teorema. Hemos dado una idea formal pero nuestro argumento tiene una serie de fallos que habrá que remediar:
 \begin{enumerate}
   \item Los espacios $(\Omega^{p,q}(M),\esc{\cdot,\cdot}_{L^2})$ no son completos y el operador $\dol$ no es acotado, de forma que tenemos una dificultad a la hora de definir el operador adjunto $\dol^*$.
   \item El espacio de las formas de tipo $(p,q)$ cerradas es de dimensión infinita, de forma que, a menos que las clases de cohomología, vistas como subespacios vectoriales sean cerradas, no tienen por qué estar determinadas por el representante de norma mínima y demostrar que esas clases son, en efecto, cerradas es algo altamente no trivial.
 \end{enumerate}

 Antes de continuar, veamos por qué, en efecto, podemos entender $\Delta$ como un genuino \emph{laplaciano}. Si $\alpha \in \Omega^{p,q}(M)$ en coordenadas locales se escribe en la forma
 \begin{equation*}
   \alpha = \sum_{|I|=p, |J|=q} \alpha_{IJ} dz_I \wedge d\bar{z}_J,
 \end{equation*}
 entonces es fácil comprobar que
 \begin{equation*}
   \Delta \alpha = -2 \sum_{|I|=p, |J|=q,i} \frac{\partial^2 \alpha_{IJ}}{\partial z_i \partial \bar{z}_i} dz_I \wedge d\bar{z}_J= -\frac{1}{2}\sum_{|I|=p, |J|=q,i}\left( \frac{\partial^2}{\partial x_i^2} + \frac{\partial^2}{\partial y_i^2} \right) \alpha_{IJ} dz_I \wedge d\bar{z}_J,
 \end{equation*}
 con $z_j=x_j+iy_j$.
 Es decir, localmente $\Delta$ actúa como el laplaciano sobre las funciones $\alpha_{IJ}$. Así, podemos pensar que el problema de buscar las formas armónicas en las clases de cohomología de $H^{p,q}(M)$ es similar al de resolver la ecuación de Laplace. Esto, junto con las observaciones anteriores, deja bastante claro que si queremos atacar la demostración del teorema de Hodge va a ser necesario recurrir a técnicas de análisis funcional, que repasaremos en la siguiente sección.

 \section{La teoría local: Nociones de análisis funcional}
 \subsection{Espacios de Hilbert}
 En primer lugar, recordemos algunas definiciones y resultados básicos sobre espacios de Hilbert.
 \begin{defn}
   Un \emph{espacio de Hilbert (complejo)}  es un espacio vectorial complejo equipado de un producto hermítico $\esc{\cdot,\cdot}$ que es completo con respecto a la métrica inducida por este producto.
 \end{defn}
 \begin{ejemplo}
   El ejemplo más importante de espacio de Hilbert complejo de dimensión infinita es $L^2(U)$, con $U\subset \RR^n$ un conjunto abierto, conexo y acotado. Este espacio está definido\footnote{Técnicamente, los elementos de $L^2(U)$ son clases de equivalencia de funciones de cuadrado integrable, donde dos funciones se consideran equivalentes si sus valores difieren tan solo en un conjunto de medida nula.} por
   \begin{equation*}
     L^2(U)= \left\{ f:U \rightarrow \CC : \int_U f^2 < \infty \right\}.
   \end{equation*}
   El producto hermítico viene dado por
   \begin{equation*}
     \esc{f,g}_{L^2(U)}=\int_U \bar{f} g.
   \end{equation*}
   \qed
 \end{ejemplo}
 \begin{defn}
   Sean $X$ e $Y$ dos espacios de Hilbert. Un operador lineal $T:X\rightarrow Y$ se dice \emph{acotado} si existe una constante $C$ tal que
   \begin{equation*}
     \norm{Tx}_Y \leq C \norm{Tx}_X
   \end{equation*}
   para todo $x\in X$. Es fácil comprobar que un operador es acotado si y sólo si es continuo.

   Un operador lineal acotado $L:X\rightarrow \CC$ se llama un \emph{funcional lineal acotado en $X$}. Definimos el \emph{espacio dual (topológico) de $X$} como el espacio $X^*$ formado por los funcionales lineales acotados en $X$. 

   Si $T:X\rightarrow Y$ es un isomorfismo lineal acotado y con inversa acotada\footnote{El teorema de la aplicación abierta garantiza que basta pedir que $T$ sea acotado y un isomorfismo lineal para que su inversa sea acotada} (esto es, $T$ es un isomorfismo lineal y un homeomorfismo), decimos que $T$ es un \emph{isomorfismo de espacios de Hilbert}.
 \end{defn}
 \begin{thm}[Teorema de representación de Riesz]
   Para cada funcional lineal acotado $L\in X^*$, existe un único elemento $x_L \in X$ tal que
   \begin{equation*}
     L(y) = \esc{x_L,y}
   \end{equation*}
   para todo $y\in X$. La aplicación 
   \begin{align*}
      X^*&\longrightarrow X\\ 
       L &\longmapsto x_L 
     \end{align*}
     es un isomorfismo.
 \end{thm}
 \begin{defn}
   Si $T:X\rightarrow Y$ es un operador lineal acotado, definimos su \emph{adjunto} $T^*:Y\rightarrow X$ como aquel que cumple
   \begin{equation*}
     \esc{Tx,y}_Y=\esc{x,T^*y}_X,
   \end{equation*}
   para todo $x \in X$ y para todo $y\in Y$. Decimos que $T:X\rightarrow X$ es \emph{autoadjunto} si $T^*=T$.
 \end{defn}
 \begin{defn}
   Un operador lineal acotado $K:X\rightarrow Y$ se dice \emph{compacto} si, para cada sucesión acotada $\left\{ x_k \right\}_{k=1}^\infty$ en $X$ existe una subsucesión $\left\{ x_{k_j} \right\}_{j=1}^\infty$ tal que $\left\{ Kx_{k_j} \right\}_{j=1}^\infty$ converge en $Y$.
 \end{defn}
 \begin{defn}
   Sea $T:X\rightarrow X$ un operador lineal acotado. Se define el \emph{espectro de $T$} como el conjunto
   \begin{equation*}
     \sigma(T)=\left\{ \lambda \in \CC: (T-\lambda I) \text{ no es biyectiva} \right\},
   \end{equation*}
   con $I$ el operador identidad.
   Un elemento $\lambda \in \sigma(T)$ se dice un \emph{autovalor de $T$} si $\ker(T-\lambda I)\neq \left\{ 0 \right\}$. El conjunto de los autovalores de $T$ se llama el \emph{espectro puntual de $T$} y se denota por $\sigma_p(T)$. Dado $\lambda \in \sigma_p(T)$, un elemento $x\in \ker(T-\lambda I)$ es un \emph{autovector de $T$ con autovalor $\lambda$}. Es decir, si $x\in X$ es un autovalor de $T$ con autovalor $\lambda$, entonces
   \begin{equation*}
     Tx = \lambda x.
   \end{equation*}
 \end{defn}
 \begin{defn}
   Un espacio de Hilbert es \emph{separable} si lo es como espacio topológico, es decir, si tiene un subconjunto denso numerable.
 \end{defn}

 El teorema central en toda esta parte es el siguiente:
 \begin{thm}[Teorema espectral]
   Sea $X$ un espacio de Hilbert separable y $K:X\rightarrow X$ un operador lineal compacto y autoadjunto. Entonces 
   \begin{enumerate}
     \item $0 \in \sigma(K)$,
     \item $\sigma(K) \setminus \left\{ 0 \right\} = \sigma_p(K) \setminus \left\{ 0 \right\}$,
     \item $\sigma(K)$ es finito o es una sucesión que tiende a cero,
     \item existe una base ortonormal de $X$ que consiste en autovectores de $K$, es decir, podemos dar una descomposición
       \begin{equation*}
	 X=\bigoplus_{i=1}^\infty E(\lambda_i),
       \end{equation*}
       con $\sigma_p(K)=\left\{ \lambda_i: i\in \mathbb{N} \right\}$ y $E(\lambda_i)$ subespacio de dimensión finita formado por los autovectores de autovalor $\lambda_i$.
   \end{enumerate}
 \end{thm}

 Los resultados de esta sección pueden leerse en \cite[Apéndice D]{evans} o en cualquier libro típico de análisis funcional, por ejemplo \cite{limaye}.

 \subsection{Espacios de Sobolev}
\begin{ejemplo}\label{ejemplo1}
  Supongamos que se nos da un problema de contorno, como por ejemplo
  \begin{equation}
    \begin{cases}
      -y''+2y=e^x+\cos x \\
      y(0)=1, y(1)=2.
    \end{cases}
    \label{eq:ejemplo1}
  \end{equation}
  Es fácil definir lo que entendemos como una solución del problema~\eqref{eq:ejemplo1}: una función $y\in C^2([0,1])$ que cumpla la ecuación y los datos. Sin embargo, podemos cambiar la función a la derecha de la ecuación por otra con una forma más complicada, por ejemplo, que no sea continua, como
  \begin{equation*}
    f(x)=
    \begin{cases}
      1, & x \in [0,1/2) \\
      0, & x \in [1/2,1).
    \end{cases}
  \end{equation*}
  De modo que ahora queremos resolver el problema de contorno
  \begin{equation}
    \begin{cases}
      -y''+2y=f(x) \\
      y(0)=1, y(1)=2.
    \end{cases}
    \label{eq:ejemplo2}
  \end{equation}
  En este caso ya no podemos pedir que la solución sea $C^2$, en esta sección vamos a tratar de ver en qué espacios pueden vivir estas funciones que podemos entender como «soluciones» de problemas como el~\eqref{eq:ejemplo2}. 

  En primer lugar, vamos a cambiar ligeramente el aspecto de nuestra ecuación. Para ello consideramos una función $\varphi$ en $[0,1]$, con $\varphi(0)=\varphi(1)=0$ tan buena como queramos, por ejemplo $\varphi\in C^{\infty}([0,1])$ y la multiplicamos por la ecuación
  \begin{equation*}
    -y''\varphi + 2y \varphi = f \varphi.
  \end{equation*}
  Integramos a ambos lados y obtenemos
  \begin{equation*}
    -\int_0^1 y'' \varphi + \int_0^1 2y \varphi = \int_0^1 f \varphi.
  \end{equation*}
  Integrando por partes
  \begin{equation*}
    \int_0^1 y' \varphi' + 2 \int_0^1 y \varphi = \int_0^1 f \varphi.
  \end{equation*}
  Buscaremos entonces qué funciones $y$ pueden hacer que estas integrales tengan sentido. Los espacios en los que viven estas $y$ serán los que llamaremos \emph{espacios de Sobolev}. 
  Para que la integral $\int_0^1 y \varphi$ tenga sentido basta pedir que $y\in L^2([0,1])$. Sin embargo, ¿qué le tenemos que pedir a $y$ para que $\int_0^1 y'\varphi'$ tenga sentido? ¿Qué es $y'$? ¿Se puede definir algún tipo de \emph{derivada}?

  Más en general, si para una función $\varphi$, denotamos por $\varphi_{x_i}$ a la derivada parcial $\frac{\partial \varphi}{\partial x_i}$, dada una función $u$, queremos hallar una función «derivada débil» (respecto de $x_i$), que podremos denotar como $u_{x_i}$ que cumpla la fórmula de integración por partes. Es decir, $u_{x_i}$ ha de ser tal que
  \begin{equation*}
    \int u \varphi_{x_i}=-\int u_{x_i} \varphi,
  \end{equation*}
  para cualquier $\varphi$ «suficientemente buena» y de soporte compacto. De nuevo, para que esta integral tenga sentido, es necesario que esta $u_{x_i}\in L^2([0,1])$. \qed
\end{ejemplo}

\begin{defn}[Derivada débil y espacios de Sobolev]
 Sea $U\subset \RR^n$ un subconjunto abierto y conexo y sea $u\in L^2(U)$. Se dice que una función $v\in L^2(U)$ es la \emph{derivada débil de $u$ respecto de $x_i$} si 
  \begin{equation*}
    \int_U u \varphi_{x_i}= -\int_{U} v \varphi
  \end{equation*}
  para toda $\varphi\in C^{\infty}(U)$ con soporte compacto en $U$. Esta $v$ se denota como $u_{x_i}$. 

  Más generalmente, para un multiíndice $\alpha=(\alpha_{i_1},\dots,\alpha_{i_p})$ con $1\leq i_1 <\cdots < i_r \leq N$, se dice que una función $v\in L^2(U)$ es una \emph{derivada débil de $u$ respecto del multiíndice $\alpha$}  si
  \begin{equation*}
    \int_{\Omega} u D^{\alpha} \varphi = (-1)^{|\alpha|} \int_{\Omega} v \varphi
  \end{equation*}
  para toda $\varphi \in C^{\infty}_c(\Omega)$, donde 
  \begin{equation*}
    D^{\alpha}=\frac{\partial^{|\alpha|}}{\partial x_{i_1}^{\alpha_{i_1}}\cdots \partial x_{i_r}^{\alpha_{i_r}}}.
  \end{equation*}
  Esta $v$ se denota como $D^\alpha u$.

  Finalmente, se definen los \emph{espacios de Sobolev} como los conjuntos
  \begin{equation*}
    H^{k}(U)=\left\{ u \in L^2(U): D^j u \in L^2(U), \text{ para todo } |j|\leq k  \right\}.
  \end{equation*}
\end{defn}

\begin{ejemplo}
  Sea $f(x)=|x|$ para $x\in (-1,1)$. Definimos
  \begin{equation*}
    f'(x)=
    \begin{cases}
      1 & x >0, \\
      -1 & x<0.
    \end{cases}
  \end{equation*}
  Veamos que, en efecto, $f'$ es la derivada débil de $f$. Basta comprobar la fórmula: dada $\varphi \in C^{\infty}_c((0,1))$, tenemos que
  \begin{align*}
    \int_{-1}^1 f(x) \varphi'(x) dx & = \int_{-1}^1 |x| \varphi'(x) dx = \int_{-1}^0 -x\varphi'(x) dx + \int_0^1 x \varphi'(x) dx \\ & = \int_{-1}^0 \varphi(x) dx - \int_0^1 \varphi(x) dx = -\int_{-1}^1 f'(x) \varphi(x) dx.
  \end{align*}
  \qed
\end{ejemplo}

  En $H^k(U)$, podemos definir la métrica hermítica
  \begin{equation*}
    \esc{u,v}_{H^k(U)}=\sum_{|\alpha|\leq k} \esc{D^\alpha u, D^\alpha v}_{L^2(U)},   
  \end{equation*}
  en particular
  \begin{equation*}
    \esc{u,v}_{H^1(U)}=\esc{u,v}_{L^2(U)}+\esc{\nabla u, \nabla v}_{L^2(U)}.
  \end{equation*}
  Esta métrica dota a los espacios $H^k(U)$ de la estructura de espacio prehilbertiano. Puede probarse, que, de hecho, estos espacios son completos, de modo que los $H^k(U)$ son espacios de Hilbert.

  Denotamos por $C^\infty_c(U)$ a todas las funciones diferenciables en $U$ con soporte compacto. Claramente, $C^\infty_c(U) \subset H^k_0(U)$ para todo $k\in \NN$. Definimos $H^k_0(U)$ como la adherencia de $C^\infty_c(U)$ en $H^k(U)$. Este espacio $H^k_0(U)$ es completo por ser cerrado y hereda la métrica de $H^k(U)$, de modo que es un espacio de Hilbert.

  Enunciamos ahora dos teoremas fundamentales sobre espacios de Sobolev que nos serán de utilidad:
  \begin{thm}[Inclusiones de Sobolev]
    Se tiene la siguiente inclusión
    \begin{equation*}
      H^{\left[ \frac{n}{2} \right]+1+k}(U) \subset C^k(U)
    \end{equation*}
    En consecuencia
    \begin{equation*}
      \bigcap_{k} H^k(U) \subset C^{\infty}(U).
    \end{equation*}
  \end{thm}
  \begin{thm}[Rellich-Kondrachov]
    El espacio $H^k(U)$ está \emph{contenido de forma compacta} en $L^2(U)$, es decir, se tiene una inclusión
$i:	H^k(U) \hookrightarrow L^2(U)$,
    con $i$ operador compacto.
  \end{thm}

  Los resultados de esta sección pueden verse expuestos con detalle en \cite{evans}. Sin embargo, recomiendo más la lectura de \cite[Capítulo 4]{wells} o de \cite{reiterschuster}, que usan las propiedades de la transformada de Fourier para dar pruebas sencillas de estos resultados.

  \section{La teoría global}
  \subsection{Espacios de Sobolev en variedades compactas}
  Volvemos al contexto original: $M$ es una variedad compleja conexa y compacta de dimensión compleja $n$. Consideremos el fibrado $\Lambda^{p,q}(T^*M)$ y $\left\{ U_\alpha, \varphi_\alpha \right\}$ una trivialización suya. Sea también $\left\{ \rho_\alpha \right\}$ una partición diferenciable de la unidad subordinada a $\left\{ U_\alpha \right\}$. Tenemos el diagrama
  \begin{center}
    \begin{tikzcd}
      \Lambda^{p,q}(T^*U_\alpha)      \arrow{rr}{\varphi_\alpha}\arrow{dd}[anchor=east]{} && \tilde{U}_\alpha \times \CC^r\arrow{dd}[anchor=west]{\pr_1} \\ 
       && \\
       U_\alpha\arrow{rr}[anchor=south]{\tilde{\varphi}_\alpha} && \tilde{U}_\alpha,
     \end{tikzcd}
   \end{center}
   de modo que tenemos un isomorfismo
   \begin{equation*}
     \tilde{\varphi}_\alpha^*:\Omega^{p,q}(U_\alpha) \rightarrow (C^\infty(\tilde{U}_{\alpha}))^r.
   \end{equation*}
   Definimos la \emph{$k$-métrica de Sobolev} en $\Omega^{p,q}(M)$ como la métrica hermítica
   \begin{equation*}
     \esc{\omega,\sigma}_{H^k}=\sum_\alpha \sum_{i=1}^r \esc{\pr_i (\tilde{\varphi}_\alpha^* \rho_\alpha \omega),\pr_i(\tilde{\varphi}_\alpha^* \rho_\alpha \sigma)}_{H^k(U_\alpha)},
   \end{equation*}
   con $\pr_i: (C^\infty(\tilde{U}_\alpha))^r \rightarrow C^\infty(\tilde{U}_\alpha)$ la $i$-ésima proyección. 
   Denotamos por $\Omega^{p,q}_{H^k}(M)$ la compleción de $\Omega^{p,q}(M)$ respecto de la métrica $\esc{\cdot,\cdot}_{H^k}$. Análogamente, denotamos $\Omega^{p,q}_{L^2}(M)$ a la compleción de $\Omega^{p,q}(M)$ respecto de la métrica $\esc{\cdot,\cdot}_{L^2}$, definida en la sección 1. Así, los espacios $(\Omega^{p,q}_{L^2}(M),\esc{\cdot,\cdot}_{L^2})$ y $(\Omega^{p,q}_{H^k}(M),\esc{\cdot,\cdot}_{H^k})$ son espacios de Hilbert.
   
   Los resultados fundamentales de la teoría local se generalizan a nuestro caso, de modo que tenemos los siguientes teoremas:
   \begin{thm}[Inclusión global de Sobolev]
     Se tiene la siguiente inclusión
     \begin{equation*}
     \bigcap_k\Omega^{p,q}_{H^k}(M)\subset \Omega^{p,q}(M)
     \end{equation*}
   \end{thm}
   \begin{thm}[Rellich-Kondrachov, versión global]
  Se tiene la siguiente inclusión compacta  $$\Omega_{H^k}^{p,q}(M) \subset \Omega_{L^2}^{p,q}(M).$$ 
   \end{thm}

   \subsection{El producto de Dirichlet}
   Volviendo a lo que estábamos tratando en la primera sección, recordemos que uno de nuestros problemas era que en general el operador $\dol$ no es acotado. La idea entonces es encontrar una métrica en $\Omega^{p,q}(M)$ en la que $\dol$ pueda ser acotado. Para ello definimos el \emph{producto de Dirichlet}:
   \begin{equation*}
     \esc{\alpha,\beta}_{D} = \esc{\alpha,\beta}_{L^2} + \esc{\dol \alpha, \dol \beta}_{L^2} + \esc{\dol^*\alpha, \dol^*\beta}_{L^2}.
   \end{equation*}
   La clave ahora es que localmente los operadores $\dol$ y $\dol^*$ actúan como $\partial_{\bar{z}}$ y $\partial_{z}$, respectivamente, de modo que no es difícil convencerse de que esta métrica es equivalente a la $1$-métrica de Sobolev $\esc{\cdot,\cdot}_{H^1}$. Así, si completamos $\Omega^{p,q}(M)$ respecto de la métrica $\esc{\cdot,\cdot}_D$ obtenemos precisamente $\Omega^{p,q}_{H^1}(M)$, de modo que usaremos indistintamente la notación $\esc{\cdot,\cdot}_{H^1}$ para la métrica $\esc{\cdot,\cdot}_D$. Ahora,
   \begin{equation*}
     \norm{\dol \omega}_{L^2}^2 = \esc{\dol \omega,\dol \omega}_{L^2} \leq \esc{\dol \omega, \dol \omega}_{L^2} + \esc{\dol^* \omega, \dol^*\omega}_{L^2} + \esc{\omega, \omega}_{L^2} = \esc{\omega,\omega}_{H^1}.
   \end{equation*}
   Por tanto, podemos extender por densidad $\dol$ a un operador que sí es acotado
   \begin{center}
     \begin{tikzcd}
       \Omega^{p,q}(M)       \arrow{rr}{\dol}\arrow[hook]{d}[anchor=east]{} && \Omega^{p,q+1}(M)\arrow[hook]{d}[anchor=west]{} \\ 
	\Omega^{p,q}_{H^1}(M)\arrow{rr}[anchor=south]{\dol} && \Omega^{p,q+1}_{L^2}(M).
      \end{tikzcd}
    \end{center}
    Análogamente, podemos extender $\dol^*$ a $\dol^*:\Omega^{p,q}_{H^1}(M) \rightarrow \Omega^{p,q}_{L^2}(M)$ y se cumple que
    \begin{equation*}
      \esc{\dol \alpha, \beta}_{L^2}= \esc{\alpha, \dol^*\beta}_{L^2}
    \end{equation*}
    para todo $\alpha \in \Omega^{p,q}_{H^1}(M)$ y para todo $\beta \in \Omega^{p,q+1}_{H^1}(M)$.

    \subsection{La descomposición espectral}
    Recordemos que habíamos reducido nuestro problema original a buscar soluciones de la ecuación de Laplace, es decir, a encontrar formas $\alpha$ tales que $\Delta \alpha =0$. Esto es lo mismo que hallar formas $\alpha$ con
    \begin{equation*}
      (I + \Delta) \alpha =\alpha.
    \end{equation*}
    El operador $T=I+\Delta$ es (al menos débilmente) invertible y además, una forma es un autovector de $T^{-1}$ si y sólo si lo es de $\Delta$. En efecto, si existe $\alpha\neq 0$ tal que $T^{-1}\alpha = \mu \alpha$, con $\mu \neq 0$ (esto es seguro porque $T$ es invertible), entonces $(I+\Delta)\mu \alpha = \alpha$, luego
    \begin{equation*}
      \Delta \alpha = \frac{1-\mu}{\mu}\alpha.
    \end{equation*}
    Por tanto, una forma $\alpha$ es un autovector de $\Delta$ con autovalor $\lambda$ si y sólo si es un autovector de $T^{-1}$ con autovalor $\frac{1}{1+\lambda}$. Tenemos entonces que nuestro problema se reduce a estudiar el espectro de $T^{-1}$. En particular, $\ker \Delta$ será precisamente el espacio $E(1)$ de autovectores de $T$ con autovalor $1$.

    En primer lugar, busquemos ese operador «inverso» de $T$. Para ello, dada una forma $\alpha\in \Omega^{p,q}_{L^2}(M)$ buscamos una forma $\beta$ tal que $T\beta=\alpha$, al menos formalmente. Ahora, lo que buscamos son soluciones débiles, para ello, consideramos una forma diferenciable $\omega\in\Omega^{p,q}(M) $, de modo que
    \begin{equation*}
      \esc{T\beta,\omega}_{L^2}=\esc{\alpha,\omega}_{L^2}.
    \end{equation*}
    Pero, como $T$ es autoadjunto, ya que $\Delta$ e $I$ lo son, realmente lo que buscamos es una $\beta$ que verifique la igualdad
    \begin{equation*}
      \esc{\beta,T\omega}_{L^2}=\esc{\alpha,\omega}_{L^2},
    \end{equation*}
    que sí que tiene sentido, no solo formalmente. Nótese además que, si $\beta \in \Omega^{p,q}_{H^1}(M)$ y $\omega \in \Omega^{p,q}(M)$, entonces
    \begin{align*}
      \esc{\beta,\omega}_{H^1}&=\esc{\beta,\omega}_{L^2}+\esc{\dol \beta,\dol \omega}_{L^2}+ \esc{\dol^*\beta,\dol^*\beta}_{L^2}\\ 
      &=\esc{\beta,\omega}_{L^2} + \esc{\beta,\dol^*\dol \omega}_{L^2}+ \esc{\beta, \dol \dol^* \omega}_{L^2} \\
      &= \esc{\beta,(I+\Delta)\omega}_{L^2}=\esc{\beta,T\omega}_{L^2}.
    \end{align*}

    \begin{lema}
      Sea $\alpha \in \Omega^{p,q}_{L^2}(M)$.
      \begin{enumerate}
	\item Existe una única $T^{-1}\alpha \in \Omega^{p,q}_{H^1}(M)$ tal que
	  \begin{equation*}
	    \esc{\alpha,\omega}_{L^2}=\esc{T^{-1}\alpha,\omega}_{H^1}=\esc{T^{-1}\alpha,T\omega}_{L^2},
	  \end{equation*}
	  para toda $\omega \in \Omega^{p,q}(M)$.
	\item El operador lineal
	  \begin{center}
	    \begin{tikzcd}
	      \Omega^{p,q}_{L^2}(M) \arrow{r}{T^{-1}}\arrow[bend left]{rr}{T^{-1}} & \Omega^{p,q}_{H^1}(M) \arrow[hook]{r} & \Omega^{p,q}_{L^2}(M)
	    \end{tikzcd}
	  \end{center}
	  es acotado, compacto y autoadjunto.
      \end{enumerate}
    \end{lema}
    \begin{proof}
 El funcional
 \begin{align*}
   L_\alpha :\Omega^{p,q}(M)&\longrightarrow \CC\\ 
   \omega &\longmapsto \esc{\alpha,\omega}_{L^2} 
   \end{align*}
   verifica
   \begin{equation*}
     |\esc{\alpha,\omega}_{L^2}|\leq \norm{\alpha}_{L^2}\norm{\omega}_{L^2} \leq \norm{\alpha}_{L^2} \norm{\omega}_{H^1},
   \end{equation*}
   de modo que se extiende por densidad a un funcional
 \begin{align*}
   L_\alpha :\Omega_{H^1}^{p,q}(M)&\longrightarrow \CC\\ 
   \omega &\longmapsto \esc{\alpha,\omega}_{L^2} .
   \end{align*}
   Como $\Omega_{H^1}^{p,q}(M)$ es un espacio de Hilbert, el teorema de representación de Riesz garantiza que existe una única $\beta \in \Omega^{p,q}_{H^1}(M)$ tal que 
   \begin{equation*}
     \esc{\beta,\omega}_{H^1} = L_\alpha(\omega)=\esc{\alpha,\omega}_{L^2},
   \end{equation*}
   para toda $\omega \in \Omega^{p,q}_{H^1}(M)$. 

   Ahora, puedo definir el operador
   \begin{align*}
     T^{-1} :\Omega^{p,q}_{L^2}(M)&\longrightarrow \Omega^{p,q}_{H^1}(M)\\ 
       \alpha &\longmapsto \beta, 
     \end{align*}
     que es autoadjunto, por serlo $T$. Además, $T^{-1}$ es acotado; en efecto,
     \begin{equation*}
       \norm{T^{-1}\alpha}_{H^1}^2=\esc{T^{-1}\alpha,T^{-1}\alpha}_{H^1}=\esc{\alpha,T^{-1}\alpha}_{L^2} \leq \norm{\alpha}_{L^2} \norm{T^{-1}\alpha}_{L^2};
     \end{equation*}
ahora, por la desigualdad de Young generalizada puedo tomar un $\eps <1$ tal que
     \begin{equation*}
       \norm{T^{-1}\alpha}_{H^1}^2\leq \norm{\alpha}_{L^2} \norm{T^{-1}\alpha}_{L^2}
       \leq \frac{1}{2} \norm{T^{-1}\alpha}^2_{L^2} + \frac{1}{2}\norm{\alpha}^2_{L^2} ;
     \end{equation*}
     finalmente, como $\norm{\cdot}_{L^2} \leq \norm{\cdot}_{H^1}$, tenemos que
     \begin{equation*}
       \norm{T^{-1}\alpha}_{H^1} \leq \norm{\alpha}^2_{H^1}.
     \end{equation*}
     Por último, como por el teorema de Rellich-Kondrachov la inclusión $\Omega^{p,q}_{H^1}(M) \hookrightarrow \Omega^{p,q}_{L^2}(M)$ es compacta, $K$ es un operador acotado y compacto y además es autoadjunto por serlo $T^{-1}$.
    \end{proof}

    En resumen, hemos construido un operador lineal $T^{-1}:\Omega^{p,q}_{L^2}(M)\rightarrow \Omega^{p,q}_{L^2}(M)$ compacto y autoadjunto al que le podemos aplicar el teorema espectral, que garantiza que su espectro $\sigma_p(T^{-1})=\left\{ \mu_i:i\in \NN \right\}$ es finito o es una sucesión que tiende a cero y que tenemos una descomposición
    \begin{equation*}
      \Omega^{p,q}_{L^2}(M)=\bigoplus_{i=1}^{\infty} E(\mu_i),
    \end{equation*}
    con $E(\mu_i)$ subespacio de dimensión finita formado por los autovectores de $T^{-1}$ de autovalor $\mu_i$. Recordemos que $\alpha$ era un autovector de $\Delta$ con autovalor $\lambda$ sólo si lo era de $T^{-1}$ con autovalor $\frac{1}{1+\lambda}$. Por tanto, $E(1)=\mathcal{H}^{p,q}_{L^2}(M)=\ker \Delta$ y tenemos la descomposición
    \begin{equation*}
      \Omega^{p,q}_{L^2}(M)=\mathcal{H}^{p,q}_{L^2}(M)\oplus(\mathcal{H}^{p,q}_{L^2}(M))^{\perp}.
    \end{equation*}

    \subsection{Regularidad}
    De momento sólo hemos obtenido soluciones débiles de la ecuación de Laplace. Es decir, todavía necesitamos garantizar que estas soluciones, en principio en $\Omega^{p,q}_{L^2}(M)$ son verdaderas formas \emph{diferenciables}, es decir, que de hecho están en $\Omega^{p,q}(M)$.

    \begin{lema}[Regularidad]
      Si $\alpha \in \Omega^{p,q}_{L^2}(M)$ es un autovector de $T^{-1}$, entonces $\alpha \in \Omega^{p,q}(M)$.
    \end{lema}
    \begin{proof}
      Como ya vimos, localmente $\Delta$ era un verdadero laplaciano, es decir, consistía en derivadas de orden 2. Por tanto, el laplaciano de una función que tenga $k$ derivadas débiles, tendrá, al menos, $k-2$ derivadas débiles, es decir, $T:\Omega^{p,q}_{H^k}(M) \rightarrow \Omega^{p,q}_{H^{k-2}}(M)$. Por tanto, $T^{-1}:\Omega^{p,q}_{H^{k}}(M) \rightarrow \Omega^{p,q}_{H^{k+2}}(M)$.     Pero, si $\alpha\in \Omega^{p,q}_{H^k}(M)$ es un autovector de $T^{-1}$ con autovalor $\mu$, entonces $$\alpha =\frac{1}{\mu} T^{-1}\alpha \in \Omega^{p,q}_{H^{k+2}}(M).$$ Iterando, tenemos que $\alpha \in \Omega^{p,q}_{H^{N}}(M)$ para todo $N\in \NN$. Por la inclusión de Sobolev,
      \begin{equation*}
	\alpha \in \bigcap_k \Omega^{p,q}_{H^k}(M) \subset \Omega^{p,q}(M),
      \end{equation*}
      como queríamos probar.
    \end{proof}
    En conclusión, podemos refinar nuestra descomposición a
    \begin{equation*}
      \Omega^{p,q}(M)=\mathcal{H}^{p,q}(M) \oplus (\mathcal{H}^{p,q}(M))^{\perp}.
    \end{equation*}

    \subsection{Operador de Green}
    En virtud de esta descomposición, tenemos que una forma $\omega \in \Omega^{p,q}(M)$ puede escribirse como 
    \begin{equation*}
      \omega=\omega^h + \omega^{\perp},
    \end{equation*}
    con $\omega^h\in \mathcal{H}^{p,q}(M)$ y $\omega^{\perp} \in (\mathcal{H}^{p,q}(M))^{\perp}$. Ahora, el operador $\Delta$ es acotado y, como la sucesión $\left\{ \mu_i \right\}_{i\in \NN}$ de los autovalores de $T^{-1}$ era decreciente a cero, la sucesión $\left\{ \lambda_i=\frac{1-\mu_i}{\mu_i} \right\}_{i\in \NN}$ de los autovalores de $\Delta$ es una sucesión creciente y podemos tomar su elemento más pequeño no nulo, $\lambda_1$, sin pérdida de generalidad. Tenemos entonces la cota
    \begin{equation*}
      \norm{\Delta \alpha}_{L^2} \geq |\lambda_1| \norm{\alpha}_{L^2},
    \end{equation*}
    en $(\mathcal{H}^{p,q}(M))^{\perp}$.
    Esto implica que $\Delta$ es invertible en $(\mathcal{H}^{p,q}(M))^{\perp}$ y su inversa es el \emph{operador de Green}
    \begin{equation*}
      G\omega = 
      \begin{cases}
	0, & \text{si } \omega \in \mathcal{H}^{p,q}(M), \\
	\frac{1}{\lambda_n}\omega, & \text{si } \omega \in E(\frac{1}{1+\lambda_n}).
      \end{cases}
    \end{equation*}
    Ahora,
    \begin{equation*}
      \Delta \omega = \Delta \omega^h + \Delta \omega^\perp = \Delta \omega^\perp
    \end{equation*}
    y, como $\omega^\perp \in (\mathcal{H}^{p,q}(M))^{\perp}$, 
    \begin{equation*}
      \omega^\perp=G\Delta \omega^\perp=G \Delta \omega.
    \end{equation*}
    Por tanto, podemos escribir
    \begin{equation*}
      \omega = \omega^h + G\Delta \omega = \omega^h + G\dol^*\dol\omega + G\dol \dol^*\omega.
    \end{equation*}
    
    Supongamos ahora que $\omega$ es $\dol$-cerrada, es decir, que $\dol \omega=0$. En tal caso
    \begin{equation*}
      \omega=\omega^h + G\dol \dol^* \omega.
    \end{equation*}
    Como $\dol$ y $\dol^*$ conmutan con $\Delta$ también lo hacen con $G$ y tenemos
    \begin{equation*}
      \omega= \omega^h + \dol(G\dol^*\omega).
    \end{equation*}
    En conclusión $[\omega]=[\omega^h]$ y la aplicación
    \begin{align*}
      \Omega^{p,q}(M)&\longrightarrow \mathcal{H}^{p,q}(M)\\ 
        \omega &\longmapsto \omega^h 
      \end{align*}
      es una proyección ortogonal que desciende a un isomorfismo
      \begin{align*}
	H^{p,q}(M)&\longrightarrow \mathcal{H}^{p,q}(M)\\ 
      [\omega] &\longmapsto \omega^h .
	\end{align*}
	Esto finaliza la demostración del teorema de Hodge.

	Un corolario inmediato de la demostración es el siguiente:
	\begin{corol}
	  Los grupos de cohomología de Dolbeaut de una variedad compleja compacta son de dimensión finita.
	\end{corol}

	\section{Dualidad de Kodaira-Serre}
	En esta sección veremos uno de las consecuencias más importantes del teorema de Hodge.
Es fácil comprobar que se verifica la igualdad $\dol^*=-*\dol *$, de modo que $*\Delta=\Delta*$. Esto implica que la estrella de Hodge induce un isomorfismo
\begin{align*}
  * :\mathcal{H}^{p,q}(M)&\longrightarrow \HH^{n-p,n-q}(M).
  \end{align*}
  En particular, como $*1=\dvol_M$, tenemos
  \begin{equation*}
    \HH^{n,n}(M) \cong \CC\cdot \dvol_M.
  \end{equation*}
  \begin{corol}[Dualidad de Kodaira-Serre]
    Se tiene que $H^n(M,\bomega^n)\cong \CC$ y que
    \begin{equation*}
      H^q(M,\bomega^p) \cong H^{n-q}(M,\bomega^{n-p}).
    \end{equation*}
  \end{corol}
  \begin{proof}
    Recordemos que el teorema de Dolbeaut (que probamos usando cohomología de haces) afirmaba que $H^q(M,\bomega^p)\cong H^{p,q}(M)$. Ahora, por el teorema de Hodge, tenemos que $H^{p,q}(M)\cong \HH^{p,q}(M)$, de modo que $$H^q(M,\bomega^p)\cong \HH^{p,q}(M).$$
    Componiendo este isomorfismo con la estrella de Hodge, obtenemos lo que queríamos probar:
    \begin{center}
      \begin{tikzcd}
	H^n(M,\bomega^n) \cong H^{n,n}(M) \cong \HH^{n,n}(M) \arrow{r}{*} & \CC	
      \end{tikzcd}
    \end{center}
    \begin{center}
      \begin{tikzcd}
	H^q(M,\bomega^p) \cong  H^{p,q}(M) \cong  \HH^{p,q}(M) \arrow{r}{*} & \HH^{n-p,n-q}(M) \cong H^{n-p,n-q}(M) \cong H^{n-q}(M,\bomega^{n-p}). 	
      \end{tikzcd}
    \end{center}
  \end{proof}
  \section{Generalizaciones}
  Sea $E$ un fibrado vectorial complejo diferenciable de rango $r$ sobre $M$ equipado de una métrica hermítica $\esc{\cdot,\cdot}_E$. Definimos un producto interno $\esc{\cdot,\cdot}_{L^2}$ en el espacio $\Gamma(M,E)$ de secciones diferenciables de $E$ mediante
  \begin{equation*}
    \esc{\xi,\eta}_{L^2}=\int_{M} \esc{\xi(x),\eta(x)}_E \dvol_M.
  \end{equation*}
  Consideremos ahora $\left\{ U_\alpha,\varphi_\alpha \right\}$ una trivialización de $E$ y $\left\{ \rho_\alpha \right\}$ una PDU subordinada a $\left\{ U_{\alpha} \right\}$. 
Tenemos el diagrama
  \begin{center}
    \begin{tikzcd}
      E|_{U_\alpha}      \arrow{rr}{\varphi_\alpha}\arrow{dd}[anchor=east]{} && \tilde{U}_\alpha \times \CC^r\arrow{dd}[anchor=west]{\pr_1} \\ 
       && \\
       U_\alpha\arrow{rr}[anchor=south]{\tilde{\varphi}_\alpha} && \tilde{U}_\alpha,
     \end{tikzcd}
   \end{center}
   de modo que tenemos un isomorfismo
   \begin{equation*}
     \tilde{\varphi}_\alpha^*:\Gamma(U_\alpha,E) \rightarrow (C^\infty(\tilde{U}_{\alpha}))^r.
   \end{equation*}
   Definimos la \emph{$k$-métrica de Sobolev} en $\Gamma(M,E)$ como la métrica hermítica
   \begin{equation*}
     \esc{\xi,\eta}_{H^k}=\sum_\alpha \sum_{i=1}^r \esc{\pr_i (\tilde{\varphi}_\alpha^* \rho_\alpha \omega),\pr_i(\tilde{\varphi}_\alpha^* \rho_\alpha \sigma)}_{H^k(U_\alpha)},
   \end{equation*}
   con $\pr_i: (C^\infty(\tilde{U}_\alpha))^r \rightarrow C^\infty(\tilde{U}_\alpha)$ la $i$-ésima proyección. 
   Denotamos por $\Gamma_{H^k}(M,E)$ la compleción de $\Gamma(M,E)$ respecto de la métrica $\esc{\cdot,\cdot}_{H^k}$. Análogamente, denotamos $\Gamma_{L^2}(M,E)$ a la compleción de $\Gamma(M,E)$ respecto de la métrica $\esc{\cdot,\cdot}_{L^2}$, definida en la sección 1. Así, los espacios $(\Gamma_{L^2}(M,E),\esc{\cdot,\cdot}_{L^2})$ y $(\Gamma_{H^k}(M,E),\esc{\cdot,\cdot}_{H^k})$ son espacios de Hilbert.
   
   De nuevo se siguen cumpliendo los resultados fundamentales de la teoría local:
   \begin{thm}[Sobolev]
     Se tiene la siguiente inclusión
     \begin{equation*}
     \bigcap_k\Gamma_{H^k}(M,E)\subset \Gamma(M,E)
     \end{equation*}
   \end{thm}
   \begin{thm}[Rellich-Kondrachov]
  Se tiene la siguiente inclusión compacta  $$\Gamma_{H^k}(M,E) \subset \Gamma_{L^2}(M,E).$$ 
   \end{thm}

   \begin{defn}
     Sean $E$ y $F$ fibrados vectoriales complejos diferenciables de rango $r$ y $s$, respectivamente, sobre $M$. Una aplicación lineal
     \begin{align*}
       L :\Gamma(M,E)&\longrightarrow \Gamma(M,F)
       \end{align*}
       es un \emph{operador diferencial de orden $k$} si para cada elección de coordenadas locales y trivializaciones locales existe un operador lineal 
	 \begin{align*}
	 \tilde{L} :(C^\infty(U))^r&\longrightarrow (C^\infty(U))^s\\
	 (f_1,\dots,f_r) &\longmapsto \left(\sum_{j=1, |\alpha|\leq k}^p a^{ij}_\alpha D^\alpha f_j\right)_{i=1,\dots,s},
	   \end{align*}
	   con las $a_{\alpha}^{ij} \in C^\infty(U)$,
	 tal que el siguiente diagrama conmuta
	 \begin{center}
	   \begin{tikzcd}
	     (C^\infty(U))^r \arrow{r}{\tilde{L}} & (C^\infty(U))^s	     \\ 
	     \Gamma(U,U\times \CC^r) \arrow{r}{} \arrow{u}{\cong} & \Gamma(U,U\times \CC^s) \arrow{u}{\cong} 	     \\ 
	     \Gamma(M,E)|_U \arrow{r}{L} \arrow[hook]{u}{} & \Gamma(M,E)|_U \arrow[hook]{u}{} 	     .
	   \end{tikzcd}
	 \end{center}	
	 Fijo $x\in M$, llamamos el \emph{símbolo de $L$} en un punto $v\in T^*_xM$ con $v\neq0$ a la aplicación lineal
	 \begin{align*}
	   \sigma(L)(x,v) :E_x&\longrightarrow F_x\\ 
	   (\xi_1,\dots,\xi_r) &\longmapsto \left( \sum_{j=1,|\alpha|=k}^p a_{\alpha}^{ij}v^\alpha\xi_j\right)_{i=1,\dots,s},
	   \end{align*}
	   donde $v^\alpha=v_{i_1}^{\alpha_1}\dots v_{i_n}^{\alpha_n}$.
	   Decimos que el operador $L$ es \emph{elíptico} si para cada $(x,v)\in T^*M$, $\sigma(L)(x,v)$ es un isomorfismo lineal. 
   \end{defn}

   El teorema clave para operadores elípticos es el siguiente:
   \begin{thm}\label{espectral}
     Sea $E$ un fibrado vectorial complejo diferenciable sobre $M$ y $T:E\rightarrow E$ un operador autoadjunto y elíptico. Entonces existen dos operadores
     \begin{align*}
       \pi:  \Gamma(M,E) &\longrightarrow \Gamma(M,E) \\ 
	 G:  \Gamma(M,E) &\longrightarrow \Gamma(M,E) 
       \end{align*}
       tales que
       \begin{enumerate}
	 \item $\pi(\Gamma(M,E))=\ker T$ y $\dim \ker T < \infty$,
	 \item $T\circ G +\pi=G\circ T + \pi = \id_{\Gamma(M,E)}$, 
	 \item $\Gamma(M,E)= \ker T \oplus G\circ T(\Gamma(M,E)) = \ker T \oplus T\circ G(\Gamma(M,E))$.
       \end{enumerate}
   \end{thm}
   \begin{defn}
     Sean $E_0,\dots,E_N$ fibrados vectoriales complejos y $L_0,\dots,L_{N-1}$ operadores diferenciales de orden $k$ en la sucesión
     \begin{center}
       \begin{tikzcd}
	 \Gamma(M,E_0)\arrow{r}{L_0} & \Gamma(M,E_1) \arrow{r}{L_1} & \Gamma(M,E_2) \arrow{r}{L_2} & \dots \arrow{r}{L_{N-1}} & \Gamma(M,E_N).
       \end{tikzcd}
     \end{center}
     Esta sucesión, que denotamos $E$, se llama un \emph{complejo de operadores} si $L_i\circ L_{i-1}=0$ para $i=1,\dots,N-1$. Si $E$ es un complejo de operadores, definimos los \emph{grupos de cohomología} del complejo
     \begin{equation*}
       H^q(E)=\frac{\ker L_q}{\mathrm{im} L_{q-1}}.
     \end{equation*}

Fijo $x\in M$, asociada a $E$ tenemos una sucesión de símbolos, para cada $v\in T^*_xM$ con $v\neq 0$,
     \begin{center}
       \begin{tikzcd}
	 0\arrow{r}	& E_{0,x}\arrow{rr}{\sigma(L_0)(x,v)} && E_{1,x} \arrow{rr}{\sigma(L_1)(x,v)}&& E_{2,x} \arrow{rr}{\sigma(L_2)(x,v)} && \dots \arrow{rr}{\sigma(L_{N-1})(x,v)} && E_{N,x} \arrow{r} & 0.
       \end{tikzcd}
     \end{center}
     Si esta sucesión de símbolos es exacta, $E$ se llama un \emph{complejo elíptico}.
   \end{defn}

   Si $E$ es un complejo elíptico, asociado a cada operador $L_j$ tenemos un operador adjunto $L_j^*:\Gamma(M,E_{j+1}) \rightarrow \Gamma(M,E_j)$ y definimos los \emph{laplacianos} del complejo elíptico 
   \begin{equation*}
     \Delta_j=L_j^*L_j + L_{j-1}L^*_{j-1}: \Gamma(M,E_j) \rightarrow \Gamma(M,E_j).
   \end{equation*}
   Se prueba fácilmente que estos laplacianos $\Delta_j$ son elípticos y autoadjuntos. Por tanto, les podemos asociar los operadores $\pi$ y $G$ del teorema \ref{espectral}. Finalmente, se obtiene el siguiente teorema
\begin{thm}[Hodge generalizado]
  Sea $E$ un complejo elíptico. Entonces:
  \begin{enumerate}
    \item Hay una descomposición ortogonal
      \begin{equation*}
	\Gamma(M,E)= \ker \Delta \oplus LL^*G\Gamma(M,E) \oplus L^*LG\Gamma(M,E),
      \end{equation*}
    \item Se cumplen las siguientes relaciones:
      \begin{enumerate}
	\item $I=\pi+\Delta G = \pi+G\Delta$,
	\item $\pi G=G\pi=\pi \Delta=\Delta \pi$,
	\item $L\Delta=\Delta L$, $L^*\Delta=\Delta L^*$,
	\item $LG=GL$, $L^*G=GL$.
      \end{enumerate}
    \item Tenemos que $\dim \ker \Delta < \infty$ y existe un isomorfismo canónico
      \begin{equation*}
	\ker \Delta_j \cong H^j(E).
      \end{equation*}
  \end{enumerate}
\end{thm}

\begin{ejemplo}
  Consideremos el complejo de de Rham (con coeficientes complejos, es decir, $\Omega^k(M)=\Omega^k(M,\CC)$)
  \begin{center}
    \begin{tikzcd}
      \Omega^0(M) \arrow{r}{d} & \Omega^1(M) \arrow{r}{d} & \cdots \arrow{r}{d} & \Omega^n(M).
    \end{tikzcd}
  \end{center}
  Nótese que $\Omega^k(M)=\Gamma(M,\Lambda^k(T^*M\otimes \CC))$ y que los operadores $d$ son operadores diferenciales. Tenemos entonces, para cada $(x,v)\in T^*M$, una sucesión de símbolos
  \begin{center}
    \begin{tikzcd}
      \Lambda^0(T_x^*M\otimes \CC) \arrow{rr}{\sigma(d)(x,v)} && \Lambda^1(T_x^*M\otimes \CC) \arrow{rr}{\sigma(d)(x,v)} && \cdots \arrow{rr}{\sigma(d)(x,v)} && \Lambda^n(T_x^*M\otimes \CC).
    \end{tikzcd}
  \end{center}
  Para cada $e\in \Lambda^k(T_x^*M\otimes \CC)$, es fácil probar que
  \begin{equation*}
    \sigma(d)(x,v)e=iv\wedge e.
  \end{equation*}
  Además, la sucesión de símbolos es exacta. Tenemos entonces que el complejo de de Rham es un complejo elíptico. En este caso los laplacianos vienen dados por
  \begin{equation*}
    \Delta=d^*d+dd^*:\Omega^k(M) \longrightarrow \Omega^k(M).
  \end{equation*}
  El teorema de Hodge generalizado nos da un isomorfismo canónico
  \begin{equation*}
    H^k_{\mathrm{dR}}(M) \cong \mathcal{H}^k(M),
  \end{equation*}
  con $\mathcal{H}^k(M)=\ker(\Delta:\Omega^k(M) \rightarrow \Omega^k(M))$.
  Además, la estrella de Hodge sigue dando un isomorfismo
  \begin{equation*}
    *:\HH^k(M) \longrightarrow \HH^{n-k}(M),
  \end{equation*}
  de modo que tenemos un isomorfismo $H_{\mathrm{dR}}^n(M) \cong \CC$ y 
  \begin{equation*}
    H_{\mathrm{dR}}^k(M) \cong H_{\mathrm{dR}}^{n-k}(M) .
  \end{equation*}
  Esto es lo que se conoce como la \emph{dualidad de Poincaré}. \qed
\end{ejemplo}
\begin{ejemplo}
  Sea $E\rightarrow M$ un fibrado vectorial holomorfo y consideremos las formas diferenciables de tipo $(p,q)$ con coeficientes en $E$
  \begin{equation*}
    \Omega^{p,q}(M,E) = \Gamma(M,\Lambda^{p,q}(T^*M\otimes \CC)\otimes E).
  \end{equation*}
  Esto da un complejo
  \begin{center}
    \begin{tikzcd}
      \cdots \arrow{r}{\dol_E} & \Omega^{p,q}(M,E) \arrow{r}{\dol_E} & \Omega^{p,q+1}(M,E) \arrow{r}{\dol_E} & \cdots,
    \end{tikzcd}
  \end{center}
  donde los operadores $\dol_E$ son operadores diferenciales. Para cada $(x,v)\in T^*M$ tenemos una sucesión de símbolos
  \begin{center}
    \begin{tikzcd}
      \cdots\arrow{rr}{\sigma(\dol_E)(x,v)} &&     \Lambda^{p,q}(T_x^*M\otimes \CC)\otimes E_x \arrow{rr}{\sigma(\dol_E)(x,v)} && \Lambda^{p,q+1}(T_x^*M\otimes \CC)\otimes E_x \arrow{rr}{\sigma(\dol_E)(x,v)} && \cdots .
    \end{tikzcd}
  \end{center}
  Para cada $f\otimes e\in \Lambda^{p,q}(T_x^*M)\otimes E_x$, si escribimos $v=v^{1,0}+v^{0,1}$, es fácil probar que
  \begin{equation*}
    \sigma(\dol_E)(x,v) f \otimes e = (iv^{0,1}\wedge f)\otimes e.
  \end{equation*}
  Además, de nuevo la sucesión de símbolos es exacta. Tenemos entonces que este complejo es elíptico y sus laplacianos vienen dados por
  \begin{equation*}
    \Delta = \dol_E^*\dol_E + \dol_E \dol_E^*:\Omega^{p,q}(M,E) \longrightarrow \Omega^{p,q}(M,E).
  \end{equation*}
  Aquí, el teorema de Hodge generalizado nos da un isomorfismo canónico
  \begin{equation*}
    H^{p,q}(M,E) \cong \HH^{p,q}(M,E), 
  \end{equation*}
  con $\HH^{p,q}(M,E)=\ker(\Delta:\Omega^{p,q}(M,E)\rightarrow \Omega^{p,q}(M,E)$. También en este caso la estrella de Hodge sigue dando un isomorfismo
  \begin{equation*}
    *:\HH^{p,q}(M,E) \longrightarrow \HH^{n-p,n-q}(M,E),
  \end{equation*}
  y además la cohomología de haces nos da un isomorfismo
  \begin{equation*}
    H^{p,q}(M,E) \cong H^q(M,\bomega^p(E)).
  \end{equation*}
  Juntando estos isomorfismos tenemos 
  \begin{equation*}
    H^q(M,\bomega^p(E))\cong H^{n-q}(M,\bomega^{n-p}(E)).
  \end{equation*}
  En particular, para $p=0$, tenemos
  \begin{equation*}
    H^q(M,\bomega^0(E))\cong H^q(M,\bomega^n(E)).
  \end{equation*}
  Si denotamos por $\mathcal{O}(E)$ las secciones holomorfas del fibrado $E$ y llamamos el \emph{fibrado canónico} a $K_M=\Lambda^n((\mathbf{T}M)^*)$ (con $\mathbf{T}M=T^{1,0}M$ el fibrado tangente holomorfo) tenemos  el siguiente isomorfismo
  \begin{equation*}
    H^k(M,\mathcal{O}(E)) \cong H^k(M,\mathcal{O}(E\otimes K_M)).
  \end{equation*}
  Este isomorfismo se conoce como la \emph{dualidad de Serre}. 
Si denotamos el \emph{haz canónico} $\omega_M=\mathcal{O}(K_M)$ y recordamos la correspondencia entre los fibrados holomorfos y los haces de $\mathcal{O}_M$~-módulos localmente libres ($E\mapsto \mathcal{O}(E)$), podemos formular la dualidad de Serre en una forma más «algebraica»:
\begin{corol}[Dualidad de Serre]
  Sea $\mathcal{E}$ un haz de $\mathcal{O}_M$-módulos localmente libre. Entonces se tiene el siguiente isomorfismo
  \begin{equation*}
    H^k(M,\mathcal{E})\cong H^k(M,\mathcal{E}\otimes \omega_M).
  \end{equation*}
\end{corol}
\qed
\end{ejemplo}

\nocite{*}
\bibliographystyle{plain}
\bibliography{biblio}
\end{document}


