%        File: hodge.tex
%     Created: mar mar 05 08:00  2019 C
% Last Change: mar mar 05 08:00  2019 C
%
\documentclass[12pt,a4paper]{article}
\usepackage[utf8]{inputenc}
\usepackage[spanish]{babel}
\usepackage{amsmath}
\usepackage{amsfonts}
\usepackage{amssymb}
\usepackage{amsthm}
\usepackage{mathtools}
\usepackage{graphicx}
\usepackage[left=2cm,right=2cm,top=2cm,bottom=2cm]{geometry}
\usepackage{tikz,tikz-cd}
\usetikzlibrary{arrows, babel}

\author{\textit{Guillermo Gallego Sánchez}}
\title{\textsc{Teoría de Hodge}}
\date{}

\newtheorem{thm}{Teorema}[section]
\newtheorem{prop}[thm]{Proposición}
\newtheorem{lema}{Lema}
\newtheorem{corol}[thm]{Corolario}
\theoremstyle{definition} \newtheorem{defn}[thm]{Definición}
\theoremstyle{definition} \newtheorem{ejemplo}[thm]{Ejemplo}
\theoremstyle{definition} \newtheorem{ejercicio}[thm]{Ejercicio}
\theoremstyle{remark} \newtheorem*{obs}{Observación}

\DeclarePairedDelimiter\norm{\lVert}{\rVert}
\DeclarePairedDelimiter\esc{\langle}{\rangle}
\newcommand{\dvol}{\mathrm{dvol}}
\newcommand{\dol}{\bar{\partial}}
\newcommand{\HH}{\mathcal{H}}
\newcommand{\RR}{\mathbb{R}}
\newcommand{\CC}{\mathbb{C}}
\newcommand{\id}{\mathrm{id}}

\newcommand{\cte}{\mathrm{cte.}}
\newcommand{\NN}{\mathbb{N}}
\newcommand{\eps}{\varepsilon}


\begin{document}
\maketitle
\section{La idea formal}
Sea $M$ una variedad compleja conexa y compacta de dimensión $n$. Recordemos que las $k$-formas diferenciables complejas admitían una descomposición
\begin{equation*}
  \Omega^k(M)=\bigoplus_{p+q=k}\Omega^{p,q}(M)
\end{equation*}
y teníamos el siguiente bicomplejo de cocadenas
\begin{center}
  \begin{tikzcd}
    &\vdots &\vdots & \\
    \cdots \arrow{r}{\bar{\partial}}  & \Omega^{p+1,q}(M) \arrow{r}{\bar{\partial}}\arrow{u}{\partial}& \Omega^{p+1,q+1}(M)\arrow{r}{\bar{\partial}}\arrow{u}{\partial} & \cdots    \\ 
    \cdots \arrow{r}{\bar{\partial}}  & \Omega^{p,q}(M) \arrow{r}{\bar{\partial}}\arrow{u}{\partial} & \Omega^{p,q+1}(M)\arrow{u}{\partial} \arrow{r}{\bar{\partial}}& \cdots   \\ 
   &\vdots \arrow{u}{\partial}&\vdots \arrow{u}{\partial}&
  \end{tikzcd}
\end{center}
de forma que definíamos la \emph{cohomología de Dolbeaut}
\begin{equation*}
  H^{p,q}(M)=\frac{\left\{  \omega \in \Omega^{p,q}(M) : \bar{\partial}\omega=0\right\}}{\bar{\partial}(\Omega^{p,q-1}(M))}.
\end{equation*}
El propósito fundamental de la \emph{teoría de Hodge} es encontrar unos representantes «especiales» de estas clases de cohomología.
  
 Aunque las conclusiones de la teoría de Hodge no son de carácter métrico, será crucial escoger de modo auxiliar una métrica hermítica $ds^2$ en $M$. En coordenadas esta métrica puede expresarse como
 \begin{equation*}
   ds^2=\sum_{i,j} h_{ij}(z) dz_i \otimes d\bar{z}_i,
 \end{equation*}
 y por Gram-Schmidt podemos «diagonalizar» esta métrica por congruencia. Es decir, podemos encontrar una referencia \emph{unitaria}, esto es, un conjunto de formas diferenciales de tipo $(1,0)$ $\left\{ \varphi_1,\dots,\varphi_n \right\}$ tal que 
 \begin{equation*}
   ds^2=\sum_i \varphi_i \otimes \bar{\varphi}_i.
 \end{equation*}
 Con esta referencia unitaria, podemos inducir una métrica en los fibrados $\Lambda^{p,q}(M)$: si 
 \begin{align*}
   \alpha_x&= \sum_{|I|=p, |J|=q} \alpha_{IJ}(x) \varphi_{i_1} \wedge \cdots \wedge \varphi_{i_p} \wedge \bar{\varphi}_{j_1} \wedge \cdots \wedge \bar{\varphi}_{j_q}|_x, \\
   \beta_x&= \sum_{|I|=p, |J|=q} \beta_{IJ}(x) \varphi_{i_1} \wedge \cdots \wedge \varphi_{i_p} \wedge \bar{\varphi}_{j_1} \wedge \cdots \wedge \bar{\varphi}_{j_q}|_x ,
 \end{align*}
  definimos
  \begin{equation*}
    \esc{\alpha_x,\beta_x}=\sum_{|I|=p, |J|=q} \overline{\alpha_{IJ}(x)} \beta_{IJ}(x)   .
  \end{equation*}
  La métrica hermítica $ds^2$ induce también una forma de volumen en $M$
  \begin{equation*}
    \dvol_M=K_n \varphi_1 \wedge \cdots \wedge \varphi_n \wedge \bar{\varphi}_1 \wedge \cdots \wedge \bar{\varphi}_n,
  \end{equation*}
  con $K_n$ cierta constante que no nos interesa.
  Definimos ahora el operador \emph{estrella de Hodge} $*:\Omega^{p,q}(M) \rightarrow \Omega^{n-p,n-q}(M)$, de forma que
  \begin{equation*}
    \alpha \wedge *\beta = \esc{\alpha,\beta} \dvol_M.
  \end{equation*}
  Usando la forma de volumen, a partir de la métrica definida en los fibrados $\esc{\cdot,\cdot}$, podemos definir una métrica «$L^2$» en los espacios de secciones $\Omega^{p,q}(M)$:
  \begin{equation*}
    \esc{\alpha,\beta}_{L^2}=\int_M \esc{\alpha,\beta} \dvol_M = \int_M \alpha \wedge *\beta.
  \end{equation*}
  Esta métrica dota a $(\Omega^{p,q}(M), \esc{\cdot,\cdot}_{L^2})$ de la estructura de espacio prehilbertiano. 

  Consideremos ahora el operador de Dolbeaut $\bar{\partial}:\Omega^{p,q}(M) \rightarrow \Omega^{p,q+1}(M)$. Este operador tiene un operador adjunto «formal» con respecto a la métrica $L^2$, $\dol^*:\Omega^{p,q+1}(M) \rightarrow \Omega^{p,q}(M)$, definido por la relación:
  \begin{equation*}
    \esc{\alpha,\dol \beta}_{L^2}=\esc{\dol^* \alpha,\beta}_{L^2},
  \end{equation*}
  para cualesquiera $\alpha \in \Omega^{p,q+1}(M)$ y $\beta \in \Omega^{p,q}(M)$.

  La idea clave ahora es la siguiente: \textit{vamos a tratar de determinar las clases de cohomología por sus elementos de norma mínima}. Es decir, nos hacemos la siguiente pregunta: dada una clase de cohomología $a\in H^{p,q}(M)$, ¿existe un representante $\alpha$ de $a$ con norma mínima?

  \begin{lema}
    Una forma $\dol$-cerrada $\alpha \in \Omega^{p,q}(M)$ es de norma mínima en $\alpha + \dol \Omega^{p,q-1}(M)$ si y sólo si $\dol^*\alpha = 0$.
  \end{lema}
  \begin{proof}
    En primer lugar, supongamos que una $\dol^*\alpha = 0$. Entonces, para toda $\beta \in \Omega^{p,q-1}(M)$ tenemos
    \begin{equation*}
      \norm{\alpha+\dol \beta}_{L^2}^2=\esc{\alpha+\dol\beta,\alpha+\dol\beta}_{L^2}=\norm{\alpha}_{L^2}^2 + \norm{\dol \beta}_{L^2}^2 + 2 \mathrm{Re} \esc{\alpha, \dol\beta}_{L^2}
    \end{equation*}
    Ahora, 
    \begin{equation*}
      \esc{\alpha,\dol \beta}_{L^2}=\esc{\dol^* \alpha, \beta}_{L^2}=0,
    \end{equation*}
    ya que $\dol^*\alpha = 0$.
    Por tanto, 
    \begin{equation*}
      \norm{\alpha+\dol \beta}_{L^2}^2=\norm{\alpha}_{L^2}^2 + \norm{\dol \beta}_{L^2}^2 \geq \norm{\alpha}_{L^2}^2.
    \end{equation*}

    Por otra parte, supongamos que $\alpha$ es de norma mínima. Entonces, para cualquier $\beta \in \Omega^{p,q-1}(M)$ ha de cumplirse que
    \begin{equation*}
      \left. \frac{\partial}{\partial t}\right|_{t=0} \norm{\alpha + t \dol \beta}^2_{L^2}=0.
    \end{equation*}
  Ahora,
  \begin{align*}
    \left. \frac{\partial}{\partial t}\right|_{t=0} \norm{\alpha + t \dol \beta}^2_{L^2}&=\left. \frac{\partial}{\partial t}\right|_{t=0} \left( \norm{\alpha}_{L^2}^2 + t^2 \norm{\beta}_{L^2}^2 + 2t \mathrm{Re}\esc{\alpha,\dol \beta}_{L^2} \right)\\ &= \left. 2t \norm{\dol \beta}^2_{L^2} + 2 \mathrm{Re}\esc{\alpha,\dol \beta}_{L^2} \right|_{t=0} = 2 \mathrm{Re}\esc{\alpha,\dol \beta}_{L^2},
  \end{align*}
  mientras que
  \begin{align*}
    \left. \frac{\partial}{\partial t}\right|_{t=0} \norm{\alpha + t \dol (i\beta)}^2_{L^2}&=\left. \frac{\partial}{\partial t}\right|_{t=0} \left( \norm{\alpha}_{L^2}^2 + t^2 \norm{\beta}_{L^2}^2 + 2t \mathrm{Im}\esc{\alpha,\dol \beta}_{L^2} \right)\\ &= \left. 2t \norm{\dol \beta}^2_{L^2} + 2 \mathrm{Im}\esc{\alpha,\dol \beta}_{L^2} \right|_{t=0} = 2 \mathrm{Im}\esc{\alpha,\dol \beta}_{L^2}.
  \end{align*}
  Tenemos entonces que $$\esc{\dol^*\alpha, \dol \beta}_{L^2}=\esc{\alpha, \dol \beta}_{L^2}=0.$$
  Como esto es cierto para cualquier $\beta \in \Omega^{p,q-1}(M)$, concluimos que $\dol^*\alpha=0$, como queríamos probar.
  \end{proof}

  De este resultado, si podemos identificar cada clase de cohomología con su elemento de norma mínima, podemos concluir, al menos formalmente, la siguiente correspondencia
  \begin{center}
    \begin{tikzcd}
      H^{p,q}(M) \arrow{r} &\left\{ \alpha \in \Omega^{p,q}(M) : \dol \alpha=0, \dol^*\alpha=0 \right\} .\arrow{l}
    \end{tikzcd}
  \end{center}
  Consideremos ahora el operador \emph{laplaciano}
  \begin{equation*}
    \Delta = \dol \dol^* + \dol^* \dol,
  \end{equation*}
  y notemos que $\Delta \alpha = 0$ si y sólo si $\dol \alpha = 0$ y $\dol^*\alpha =0$. En efecto, en un sentido es claro. Veamos entonces qué pasa si $\Delta \alpha = 0$. En tal caso, lo que sucede es que
  \begin{equation*}
    0 = \esc{\Delta \alpha, \alpha}_{L^2} = \esc{\dol \dol^* \alpha, \alpha}_{L^2} + \esc{\dol^* \dol \alpha, \alpha}_{L^2} = \esc{\dol^* \alpha, \dol^* \alpha}_{L^2} + \esc{\dol \alpha, \dol \alpha}_{L^2} = \norm{\dol^* \alpha}_{L^2} + \norm{\dol \alpha}_{L^2},
  \end{equation*}
  con lo cual, $\dol \alpha = \dol^*\alpha=0$. Definimos el conjunto de \emph{formas armónicas} de tipo $(p,q)$ como 
  \begin{equation*}
    \HH^{p,q}(M)=\left\{ \alpha \in \Omega^{p,q}(M) : \Delta \alpha=0 \right\}.   
  \end{equation*}
 Precisamente lo que queremos demostrar es lo siguiente:
 \begin{thm}[Hodge]
   Existe un isomorfismo
   \begin{center}
     \begin{tikzcd}
       H^{p,q}(M) \arrow{r}{\cong} & \HH^{p,q}(M).
     \end{tikzcd}
   \end{center}
 \end{thm}
   
 Nótese que aún no hemos probado este teorema, sino que hemos dado una idea formal, sin embargo, nuestro argumento tiene una serie de fallos que habrá que remediar:
 \begin{enumerate}
   \item Los espacios $(\Omega^{p,q}(M),\esc{\cdot,\cdot}_{L^2})$ no son completos y el operador $\dol$ no es acotado, de forma que tenemos una dificultad a la hora de definir el operador adjunto $\dol^*$.
   \item El espacio de las formas de tipo $(p,q)$ cerradas es de dimensión infinita, de forma que, a menos que las clases de cohomología, vistas como subespacios vectoriales sean cerradas, no tienen por qué estar determinadas por el representante de norma mínima y demostrar que esas clases son, en efecto, cerradas es algo altamente no trivial.
 \end{enumerate}

 Antes de continuar, veamos por qué, en efecto, podemos entender $\Delta$ como un genuino \emph{laplaciano}. Si $\alpha \in \Omega^{p,q}(M)$ en coordenadas locales se escribe en la forma
 \begin{equation*}
   \alpha = \sum_{|I|=p, |J|=q} \alpha_{IJ} dz_I \wedge d\bar{z}_J,
 \end{equation*}
 entonces es fácil comprobar que
 \begin{equation*}
   \Delta \alpha = -2 \sum_{|I|=p, |J|=q,i} \frac{\partial^2 \alpha_{IJ}}{\partial z_i \partial \bar{z}_i} dz_I \wedge d\bar{z}_J= -\frac{1}{2}\sum_{|I|=p, |J|=q,i}\left( \frac{\partial^2}{\partial x_i^2} + \frac{\partial^2}{\partial y_i^2} \right) \alpha_{IJ} dz_I \wedge d\bar{z}_J,
 \end{equation*}
 con $z_j=x_j+iy_j$.
 Es decir, localmente $\Delta$ actúa como el laplaciano sobre las funciones $\alpha_{IJ}$. Así, podemos pensar que el problema de buscar las formas armónicas en las clases de cohomología de $H^{p,q}(M)$ es similar al de resolver la ecuación de Laplace. Esto, junto con las observaciones anteriores, deja bastante claro que si queremos atacar la demostración del teorema de Hodge va a ser necesario recurrir a técnicas de análisis funcional, que repasaremos en la siguiente sección.

 \section{La teoría local: Nociones de análisis funcional}
 \subsection{Espacios de Hilbert}
 En primer lugar, recordemos algunas definiciones y resultados básicos sobre espacios de Hilbert.
 \begin{defn}
   Un \emph{espacio de Hilbert (complejo)}  es un espacio vectorial complejo equipado de un producto hermítico $\esc{\cdot,\cdot}$ que es completo con respecto a la métrica inducida por este producto.
 \end{defn}
 \begin{ejemplo}
   El ejemplo más importante de espacio de Hilbert complejo de dimensión infinita es $L^2(U)$, con $U\subset \CC^n$ un conjunto abierto, conexo y acotado. Este espacio está definido\footnote{Técnicamente, los elementos de $L^2(U)$ son clases de equivalencia de funciones de cuadrado integrable, donde dos funciones se consideran equivalentes si sus valores difieren tan solo en un conjunto de medida nula.} por
   \begin{equation*}
     L^2(U)= \left\{ f:U \rightarrow \CC : \int_U f^2 < \infty \right\}.
   \end{equation*}
   El producto hermítico viene dado por
   \begin{equation*}
     \esc{f,g}=\int_U \bar{f} g.
   \end{equation*}
 \end{ejemplo}
 \begin{defn}
   Sean $X$ e $Y$ dos espacios de Hilbert. Un operador lineal $T:X\rightarrow Y$ se dice \emph{acotado} si existe una constante $C$ tal que
   \begin{equation*}
     \norm{Tx}_Y \leq C \norm{Tx}_X
   \end{equation*}
   para todo $x\in X$. Es fácil comprobar que un operador es acotado si y sólo si es continuo.

   Un operador lineal acotado $L:X\rightarrow \CC$ se llama un \emph{funcional lineal acotado en $X$}. Definimos el \emph{espacio dual (topológico) de $X$} como el espacio $X^*$ formado por los funcionales lineales acotados en $X$. 

   Si $T:X\rightarrow Y$ es un isomorfismo lineal acotado y con inversa acotada\footnote{El teorema de la aplicación abierta garantiza que basta pedir que $T$ sea acotado y un isomorfismo lineal para que su inversa sea acotada} (esto es, $T$ es un isomorfismo lineal y un homeomorfismo), decimos que $T$ es un \emph{isomorfismo de espacios de Hilbert}.
 \end{defn}
 \begin{thm}[Teorema de representación de Riesz]
   Para cada funcional lineal acotado $L\in X^*$, existe un único elemento $x_L \in X$ tal que
   \begin{equation*}
     L(y) = \esc{x_L,y}
   \end{equation*}
   para todo $y\in X$. La aplicación 
   \begin{align*}
      X^*&\longrightarrow X\\ 
       L &\longmapsto x_L 
     \end{align*}
     es un isomorfismo.
 \end{thm}
 \begin{defn}
   Si $T:X\rightarrow X$ es un operador lineal acotado, definimos su \emph{adjunto} $T^*:X\rightarrow X$ como aquel que cumple
   \begin{equation*}
     \esc{Tx,y}=\esc{x,T^*y},
   \end{equation*}
   para todos $x,y \in X$. Decimos que $T$ es \emph{autoadjunto} si $T^*=T$.
 \end{defn}
 \begin{defn}
   Un operador lineal acotado $K:X\rightarrow Y$ se dice \emph{compacto} si, para cada sucesión acotada $\left\{ x_k \right\}_{k=1}^\infty$ en $X$ existe una subsucesión $\left\{ x_{k_j} \right\}_{j=1}^\infty$ tal que $\left\{ Kx_{k_j} \right\}_{j=1}^\infty$ converge en $Y$.
 \end{defn}
 \begin{defn}
   Sea $T:X\rightarrow X$ un operador lineal acotado. Se define el \emph{espectro de $T$} como el conjunto
   \begin{equation*}
     \sigma(T)=\left\{ \lambda \in \CC: (T-\lambda \id_X) \text{ no es biyectiva} \right\}. 
   \end{equation*}
   Un elemento $\lambda \in \sigma(T)$ se dice un \emph{autovalor de $T$} si $\ker(T-\lambda \id_X)\neq \left\{ 0 \right\}$. El conjunto de los autovalores de $T$ se llama el \emph{espectro puntual de $T$} y se denota por $\sigma_p(T)$. Dado $\lambda \in \sigma_p(T)$, un elemento $x\in \ker(T-\lambda \id_X)$ es un \emph{autovector de $T$ con autovalor $\lambda$}. Es decir, si $x\in X$ es un autovalor de $T$ con autovalor $\lambda$, entonces
   \begin{equation*}
     Tx = \lambda x.
   \end{equation*}
 \end{defn}
 \begin{defn}
   Un espacio de Hilbert es \emph{separable} si lo es como espacio topológico, es decir, si tiene un subconjunto denso numerable.
 \end{defn}

 El teorema central en toda esta parte es el siguiente:
 \begin{thm}[Teorema espectral]
   Sea $X$ un espacio de Hilbert separable y $K:X\rightarrow X$ un operador lineal compacto y autoadjunto. Entonces 
   \begin{enumerate}
     \item $0 \in \sigma(K)$,
     \item $\sigma(K) \setminus \left\{ 0 \right\} = \sigma_p(K) \setminus \left\{ 0 \right\}$,
     \item $\sigma(K)$ es finito o es una sucesión que tiende a cero,
     \item existe una base ortonormal de $X$ que consiste en autovectores de $K$, es decir, podemos dar una descomposición
       \begin{equation*}
	 X=\bigoplus_{i=1}^\infty E(\lambda_i),
       \end{equation*}
       con $\sigma_p(K)=\left\{ \lambda_i: i\in \mathbb{N} \right\}$ y $E(\lambda_i)$ el subespacio formado por los autovectores de autovalor $\lambda_i$.
   \end{enumerate}
 \end{thm}


 \subsection{Espacios de Sobolev}
\begin{ejemplo}\label{ejemplo1}
  Supongamos que se nos da un problema de contorno, como por ejemplo
  \begin{equation}
    \begin{cases}
      -y''+2y=e^x+\cos x \\
      y(0)=1, y(1)=2.
    \end{cases}
    \label{eq:ejemplo1}
  \end{equation}
  Es fácil definir lo que entendemos como una solución del problema~\eqref{eq:ejemplo1}: una función $y\in C^2([0,1])$ que cumpla la ecuación y los datos. Sin embargo, podemos cambiar la función a la derecha de la ecuación por otra con una forma más complicada, por ejemplo, que no sea continua, como
  \begin{equation*}
    f(x)=
    \begin{cases}
      1, & x \in [0,1/2) \\
      0, & x \in [1/2,1).
    \end{cases}
  \end{equation*}
  De modo que ahora queremos resolver el problema de contorno
  \begin{equation}
    \begin{cases}
      -y''+2y=f(x) \\
      y(0)=1, y(1)=2.
    \end{cases}
    \label{eq:ejemplo2}
  \end{equation}
  En este caso ya no podemos pedir que la solución sea $C^2$, en esta sección vamos a tratar de ver en qué espacios pueden vivir estas funciones que podemos entender como «soluciones» de problemas como el~\eqref{eq:ejemplo2}. 

  En primer lugar, vamos a cambiar ligeramente el aspecto de nuestra ecuación. Para ello consideramos una función $\varphi$ en $[0,1]$, con $\varphi(0)=\varphi(1)=0$ tan buena como queramos, por ejemplo $\varphi\in C^{\infty}([0,1])$ y la multiplicamos por la ecuación
  \begin{equation*}
    -y''\varphi + 2y \varphi = f \varphi.
  \end{equation*}
  Integramos a ambos lados y obtenemos
  \begin{equation*}
    -\int_0^1 y'' \varphi + \int_0^1 2y \varphi = \int_0^1 f \varphi.
  \end{equation*}
  Integrando por partes
  \begin{equation*}
    \int_0^1 y' \varphi' + 2 \int_0^1 y \varphi = \int_0^1 f \varphi.
  \end{equation*}
  Buscaremos entonces qué funciones $y$ pueden hacer que estas integrales tengan sentido. Los espacios en los que viven estas $y$ serán los que llamaremos \emph{espacios de Sobolev}. 
  Para que la integral $\int_0^1 y \varphi$ tenga sentido basta pedir que $y\in L^2([0,1])$. Sin embargo, ¿qué le tenemos que pedir a $y$ para que $\int_0^1 y'\varphi'$ tenga sentido? ¿Qué es $y'$? ¿Se puede definir algún tipo de \emph{derivada}?

  Más en general, si para una función $\varphi$, denotamos por $\varphi_{x_i}$ a la derivada parcial $\frac{\partial \varphi}{\partial x_i}$, dada una función $u$, queremos hallar una función «derivada débil» (respecto de $x_i$), que podremos denotar como $u_{x_i}$ que cumpla la fórmula de integración por partes. Es decir, $u_{x_i}$ ha de ser tal que
  \begin{equation*}
    \int u \varphi_{x_i}=-\int u_{x_i} \varphi,
  \end{equation*}
  para cualquier $\varphi$ «suficientemente buena» y de soporte compacto. De nuevo, para que esta integral tenga sentido, es necesario que esta $u_{x_i}\in L^2([0,1])$. \qed
\end{ejemplo}

\begin{defn}[Derivada débil y espacios de Sobolev]
 Sea $U\subset \CC^n$ un subconjunto abierto y conexo y sea $u\in L^2(U)$. Denotamos $(z_1=x_1+iy_1,\dots,z_n=x_n+iy_n)$ las coordenadas en $U$. Se dice que una función $v\in L^2(U)$ es la \emph{derivada débil de $u$ respecto de $x_i$} si 
  \begin{equation*}
    \int_U u \varphi_{x_i}= -\int_{U} v \varphi
  \end{equation*}
  para toda $\varphi\in C^{\infty}(U)$ con soporte compacto en $U$. Esta $v$ se denota como $u_{x_i}$. Análogamente, definimos la derivada débil $u_{y_i}$.

  Más generalmente, para dos multiíndices $\alpha=(\alpha_{i_1},\dots,\alpha_{i_p})$, $\beta=(\beta_{j_1},\dots,\beta_{j_p})$ con $1\leq i_1 <\cdots < i_r \leq N$, $1\leq j_1 <\cdots < j_r \leq N$, se dice que una función $v\in L^2(U)$ es una \emph{derivada débil de $u$ respecto de los multiíndices $\alpha$ y $\beta$}  si
  \begin{equation*}
    \int_{\Omega} u D^{\alpha,\beta} \varphi = (-1)^{|\alpha|} \int_{\Omega} v \varphi
  \end{equation*}
  para toda $\varphi \in C^{\infty}_c(\Omega)$, donde 
  \begin{equation*}
    D^{\alpha}=\frac{\partial^{|\alpha|+|\beta|}}{\partial x_{i_1}^{\alpha_{i_1}}\cdots \partial x_{i_r}^{\alpha_{i_r}}\partial y_{j_1}^{\beta_{j_1}}  \cdots \partial y_{j_n}^{\beta_{j_n}}}.
  \end{equation*}
  Esta $v$ se denota como $D^\alpha u$.

  Finalmente, se definen los \emph{espacios de Sobolev} como los conjuntos
  \begin{equation*}
    H^{k}(U)=\left\{ u \in L^2(U): D^j u \in L^2(U), \text{ para todo } |j|\leq k  \right\}.
  \end{equation*}
\end{defn}

\begin{ejemplo}
  Sea $f(x)=|x|$ para $x\in (-1,1)$. Definimos
  \begin{equation*}
    f'(x)=
    \begin{cases}
      1 & x >0, \\
      -1 & x<0.
    \end{cases}
  \end{equation*}
  Veamos que, en efecto, $f'$ es la derivada débil de $f$. Basta comprobar la fórmula: dada $\varphi \in C^{\infty}_c((0,1))$, tenemos que
  \begin{align*}
    \int_{-1}^1 f(x) \varphi'(x) dx & = \int_{-1}^1 |x| \varphi'(x) dx = \int_{-1}^0 -x\varphi'(x) dx + \int_0^1 x \varphi'(x) dx \\ & = \int_{-1}^0 \varphi(x) dx - \int_0^1 \varphi(x) dx = -\int_{-1}^1 f'(x) \varphi(x) dx.
  \end{align*}
  \qed
\end{ejemplo}

  En $H^k(U)$, podemos definir el producto escalar
  \begin{equation*}
    \esc{u,v}_{H^k(U)}=\sum_{|\alpha|\leq k} \esc{D^\alpha u, D^\alpha v}_{L^2(U)},   
  \end{equation*}
  en particular
  \begin{equation*}
    \esc{u,v}_{H^1(U)}=\esc{u,v}_{L^2(U)}+\esc{du, d v}_{L^2(U)}.
  \end{equation*}
  Este producto escalar dota a los espacios $H^k(U)$ de la estructura de espacio prehilbertiano. Puede probarse, que, de hecho, estos espacios son completos, de modo que los $H^k(U)$ son espacios de Hilbert.

  Denotamos por $C^\infty_c(U)$ a todas las funciones diferenciables en $U$ con soporte compacto. Claramente, $C^\infty_c(U) \subset H^k_0(U)$ para todo $k\in \NN$. Definimos $H^k_0(U)$ como la adherencia de $C^\infty_c(U)$ en $H^k(U)$. Este espacio $H^k_0(U)$ es completo por ser cerrado y hereda el producto escalar de $H^k(U)$, de modo que es un espacio de Hilbert.

  Enunciamos ahora dos teoremas fundamentales sobre espacios de Sobolev que nos serán de utilidad:
  \begin{thm}[Inclusiones de Sobolev]
    Se tiene la siguiente inclusión
    \begin{equation*}
      H^{\left[ \frac{n}{2} \right]+1+k}(U) \subset C^k(U)
    \end{equation*}
    En consecuencia
    \begin{equation*}
      \bigcap_{k} H^k(U) \subset C^{\infty}(U).
    \end{equation*}
  \end{thm}
  \begin{thm}[Rellich-Kondrachov]
    El espacio $H^k(U)$ está \emph{contenido de forma compacta} en $L^2(U)$, es decir, se tiene una inclusión
$i:	H^k(U) \hookrightarrow L^2(U)$,
    con $i$ operador compacto.
  \end{thm}
\end{document}


